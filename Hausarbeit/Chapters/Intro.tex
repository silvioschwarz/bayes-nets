% Chapter 1

\chapter{Introduction} % Main chapter title

\label{Chapter1} % For referencing the chapter elsewhere, use \ref{Chapter1} 

\lhead{Chapter 1. \emph{Introduction}} % This is for the header on each page - perhaps a shortened title
The following assignment is an exam for the master course "MGEW23: Einführung in Bayessche Netze für Geowissenschaftler" in the scope of the master program "Geowissenschaften" at Universität Potsdam.\\
The purpose of this work is to use Bayesian Networks in an example that resembles simplified questions one would encounter in the assessment of natural hazards. The task is to build ground motion models from synthetic data and quantify their ability to predict values of peak ground acceleration (PGA) given the input variables.\\
In the context of this paper four Bayesian Networks have been learned. These include a causal network, a naive Bayes network, a constraint-based network using a grow-shrink algorithm and a score-based one from a hill-climber. Special care is given to the evaluation of the prediction performance. A mean squared error and a log-likelihood score are computed on a testing set in order to get an estimate of the out-of-sample performance. In a next step the concept of testing on unseen data is expanded to crossvalidation and finally an attempt of a bias-variance-decomposition is made to further develop this concept and to draw conclusion where potential for improvement lies.\\
The computations were performed using R \citep{R} in combination with the IDE RStudio \citep{Rstudio}. For setting up and working with Bayesian Networks the R package bnlearn \citep{bnlearn} was used.